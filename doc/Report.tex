\documentclass[letterpaper,12pt]{article}
\usepackage{setspace}
\usepackage[utf8]{inputenc}
\usepackage[english]{babel}
\title{OARS}
\author{Satvik Chauhan}
\date{{\small \today}}

\begin{document}
\maketitle
\section{Introduction}


\subsection{Main Models}
\begin{description}
\item[Student:]
Student Model has all the basic information about the student including his
personal information. It maintains relationship with department, academic
information, registration form and template models to capture the properties
related to a student.
\item[Faculty]
Faculty Model does the same task as the student model but for faculty. It has
relationship with offered courses and departmennt. We have allowed a faculty
to take as amny courses and a course can be taken by more than one faculty.
\item[Department]
Department model is used just to keep track of existing departments. It is
linked to all the models which have the concept of department.
\item[Academic Information]
Academic information model keeps track of past performances of a student. It
is used to generate the transcript and calculate performace index (cpi,spi
etc). When an instructor submits the grade the information is moved to this
model.
\item[Course]
Course model consist of all the courses that are offered till now in the
history of the institute.
\item[Offered Course]
Offered courses are the actual courses that are offered next semester. So a
student can request courses to take in next semester only for these
courses. Offered courses are linked to courses to preserve one to one
relationship between them.
\item[Lecture Time]
Lecture time captures the weekly schedule of the offered courses. It enables
us to automatically detect timetable clashes and also generate timetable
for the offered courses taken by the student.
\item[Registration Form]
Registration form keeps track of the pre-registration. A student can add or
delete courses to this form and then submit it to DUGC for verification. It is
also used to see what all courses the student is taking this semester so to
enable a faculty to submit grades for the students.
\item[Template]
Template is used by any priviledged person to create a semester template
constraint for a student or a group of students. We can create compulsory as
well as bucket courses to be taken by a student in a particular semester and
this constraint will be checked when the student submits his pre registration
form.
\end{description}

\end{document}
