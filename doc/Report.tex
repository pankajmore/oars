\documentclass[letterpaper,12pt]{article}
\usepackage{setspace}
\usepackage[utf8]{inputenc}
\usepackage[english]{babel}
\title{OARS}
\author{Satvik Chauhan}
\date{{\small \today}}

\begin{document}
\maketitle
\section{Introduction}

\subsection{Launguage and Framework Considerations}

We chose Ruby on Rails as the underlying web framework for our
project. After working with it for the project , it seems to be much
better than other frameworks that we had previously tried. Following
are the important features that stand out in our mind:
\begin{itemize}
\item Being one of the most popular frameworks , it has a huge
  ecosystem of users and developers and support of a lot of low level
  detail are already taken care of. Lot of code is bolierplate in a
  web application. Compared to other frameworks , it seems that in
  rails least amount of boilerplate needs to be written manually.This
  leads to huge increase in productivity.
\item The idea of Convention over Configuration leads to better design
  patterns and less time spent in configuring various details.
\item DRY : It provides good support of writing modular, reusable code
  at model , views and controller layer. The facilty of writing
  partial views was very useful in combatting duplicate code in
  presentation logic. 
\item RESTful : One of the best practices that rails uses by default
  is restful architecture for models and controllers. It makes it
  easier to integrate with other ecosystems and no effort needs to be
  spent on building a restful api for the system in future.
\item TDD : Rails by default fully embraces Test Driven Developement,
  all the controllers have test specs auto generated. There are very
  well supported and extensive libraries for both blackbox and
  whitebox based testing.
\item Agile : From personal experience , ruby and rails are both
  sutable for agile methodologies. The framework and its design
  decisons feel quite natural and don't get in the way. No time is
  wasted in fighting the framework as in many other competeing
  frameworks.
\end{itemize}

We choose git for version control and github for source code hosting.
Github also provides wiki and issues to handle various project details
and management issues. The repository can be cloned from \cite{github}.

Heroku is PaaS platform for hosting web applications. A major benefit
of using rails and heroku was the ease of deployement and continuous
integration facilities available in the ecosystem. A prototype is
hosted at \cite{heroku}

 
\subsection{Main Models}
\begin{description}
\item[Student:]
Student Model has all the basic information about the student including his
personal information. It maintains relationship with department, academic
information, registration form and template models to capture the properties
related to a student.
\item[Faculty]
Faculty Model does the same task as the student model but for faculty. It has
relationship with offered courses and departmennt. We have allowed a faculty
to take as amny courses and a course can be taken by more than one faculty.
\item[Department]
Department model is used just to keep track of existing departments. It is
linked to all the models which have the concept of department.
\item[Academic Information]
Academic information model keeps track of past performances of a student. It
is used to generate the transcript and calculate performace index (cpi,spi
etc). When an instructor submits the grade the information is moved to this
model.
\item[Course]
Course model consist of all the courses that are offered till now in the
history of the institute.
\item[Offered Course]
Offered courses are the actual courses that are offered next semester. So a
student can request courses to take in next semester only for these
courses. Offered courses are linked to courses to preserve one to one
relationship between them.
\item[Lecture Time]
Lecture time captures the weekly schedule of the offered courses. It enables
us to automatically detect timetable clashes and also generate timetable
for the offered courses taken by the student.
\item[Registration Form]
Registration form keeps track of the pre-registration. A student can add or
delete courses to this form and then submit it to DUGC for verification. It is
also used to see what all courses the student is taking this semester so to
enable a faculty to submit grades for the students.
\item[Template]
Template is used by any priviledged person to create a semester template
constraint for a student or a group of students. We can create compulsory as
well as bucket courses to be taken by a student in a particular semester and
this constraint will be checked when the student submits his pre registration
form.
\end{description}


\begin{thebibliography}{9}

\bibitem{github}
  https://github.com/pankajmore/oars

\bibitem{heroku}
  oars.herokuapp.com

\end{thebibliography}
\end{document}
