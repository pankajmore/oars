\documentclass[letterpaper,12pt]{article}
\usepackage{setspace}
\usepackage[utf8]{inputenc}
\usepackage[english]{babel}
\title{OARS}
\author{Satvik Chauhan, Rahul Ajmera, Pankaj More}
\date{{\small \today}}

\begin{document}
\maketitle
\section{Introduction}
\subsection{Launguage and Framework Considerations}

We chose Ruby on Rails as the underlying web framework for our
project. After working with it for the project , it seems to be much
better than other frameworks that we had previously tried. Following
are the important features that stand out in our mind:
\begin{itemize}
\item Being one of the most popular frameworks , it has a huge
  ecosystem of users and developers and support of a lot of low level
  detail are already taken care of. Lot of code is bolierplate in a
  web application. Compared to other frameworks , it seems that in
  rails least amount of boilerplate needs to be written manually.This
  leads to huge increase in productivity.
\item The idea of Convention over Configuration leads to better design
  patterns and less time spent in configuring various details.
\item DRY : It provides good support of writing modular, reusable code
  at model , views and controller layer. The facilty of writing
  partial views was very useful in combatting duplicate code in
  presentation logic. 
\item RESTful : One of the best practices that rails uses by default
  is restful architecture for models and controllers. It makes it
  easier to integrate with other ecosystems and no effort needs to be
  spent on building a restful api for the system in future.
\item TDD : Rails by default fully embraces Test Driven Developement,
  all the controllers have test specs auto generated. There are very
  well supported and extensive libraries for both blackbox and
  whitebox based testing.
\item Agile : From personal experience , ruby and rails are both
  sutable for agile methodologies. The framework and its design
  decisons feel quite natural and don't get in the way. No time is
  wasted in fighting the framework as in many other competeing
  frameworks.
\end{itemize}

We choose git for version control and github for source code hosting.
Github also provides wiki and issues to handle various project details
and management issues. The repository can be cloned from \cite{github}.

Heroku is PaaS platform for hosting web applications. A major benefit
of using rails and heroku was the ease of deployement and continuous
integration facilities available in the ecosystem. A prototype is
hosted at \cite{heroku}



\subsection{Main Models}
\begin{description}
\item[Student:]
Student Model has all the basic information about the student including his
personal information. It maintains relationship with department, academic
information, registration form and template models to capture the properties
related to a student.
\item[Faculty]
Faculty Model does the same task as the student model but for faculty. It has
relationship with offered courses and departmennt. We have allowed a faculty
to take as amny courses and a course can be taken by more than one faculty.
\item[Department]
Department model is used just to keep track of existing departments. It is
linked to all the models which have the concept of department.
\item[Academic Information]
Academic information model keeps track of past performances of a student. It
is used to generate the transcript and calculate performace index (cpi,spi
etc). When an instructor submits the grade the information is moved to this
model.
\item[Course]
Course model consist of all the courses that are offered till now in the
history of the institute.
\item[Offered Course]
Offered courses are the actual courses that are offered next semester. So a
student can request courses to take in next semester only for these
courses. Offered courses are linked to courses to preserve one to one
relationship between them.
\item[Lecture Time]
Lecture time captures the weekly schedule of the offered courses. It enables
us to automatically detect timetable clashes and also generate timetable
for the offered courses taken by the student.
\item[Registration Form]
Registration form keeps track of the pre-registration. A student can add or
delete courses to this form and then submit it to DUGC for verification. It is
also used to see what all courses the student is taking this semester so to
enable a faculty to submit grades for the students.
\item[Template]
Template is used by any priviledged person to create a semester template
constraint for a student or a group of students. We can create compulsory as
well as bucket courses to be taken by a student in a particular semester and
this constraint will be checked when the student submits his pre registration
form.
\end{description}
% Gaurav's part start
\subsection{Features:}
The various features implemented in the software are listed below:
\begin{description}
\item[Basic Sign-up and Sign-in]
There is a sign up and sign in facility for both student and faculty which
tracks there profile and creates sessions for secure browsing.
\item[Pre-Registration]
 In pre-registration section we have implemented course (offered course) request
 which can be accepted or rejected by the faculties who are taking the courses.
 The course can only be requested if pre-requsites are satisfied. Faculty can send
 meet message to student keeping the request on hold.
 Once the courses are accepted they can be added to pre-registration. If there is any 
 time-conflict with already added courses then the course cannot be added.
 If the form does not satisfy the semester template applied to him his form can't be submmited.
 After the submission of the form it can be withdrawn before dugc accepts.
 \item[Time-Table]
 Time table can be seen once any course is added to the form. It automatically alerts if any 
 clashibg course is added to pre-registration form.
 \item[Course Constraints \& Templates]
 This facility provides dugc to create or modify course templates. Each template contains 
 several course constraint which contains courses compulsory for student to take. One constraint
 can contain more than one course which means that the student has to do one course out of the given courses.
 \item[Grade Submission \& Transcript]
 Grades can be submitted by any faculties taking that
 course. Once the grades are submitted they are reflected in the transcript provided to the student.
 \item[Transfer Role]
Dugc can tranfer his role to any other faculty. This power can be revoked by dugc once the requirement
is complete. During this period dugc's facilities are accessible to both the faculty.
\end{description}

\section{Software Testing and Agile Software Development}
We developed our OARS using iterartive and incremental development based on the developed test cases prior to the development of actual code. We designed automated test cases for every unit (class) that needs to be added for our functionalities, than the developed code was tested upon the test cases, if the test cases passes we would combine the new unit with our previous model and continue with the same process for futher development, else we would rectify our problem and iterate the whole preocess again. The test cases were written. We wrote our test cases using \emph{Cucumber} which is a tool to run test cases in ruby, written in \emph{BDD}  Behavior Driven Development style. Each test case consist of a feature and multiple scenarios, where feature defines the testing area and scenarios defines the multiple paths which the test case could take. The whole test-case consists of two files:
\begin{itemize}
\item \emph{Feature File}: This is a normal file written with BDD style (close to Natural Language) which contains basically three commands \emph{Given},\emph{When} and \emph{Then} which defines the complete scenario.
\item \emph{Step Defination}: This file contains the actual code that needs to be run while going through the steps mentioned in the feature file. The steps in the feature file act as a regular expression in the feature definations and the code written in the function defination gets executed while cucumber parses to the feature file. Basically what it does is that it visits throuhout the application checking the correctness in terms of the view(output) of the application, by browsing though the \emph{HTML} code, filling input values and navigating from one location to another in the application. The step definitions are written in Ruby. This gives much flexibility on how the test steps are executed. It can use an already exisiting step definitions for a given step if you need the same step for checking some other functionality.
\end{itemize} 
These type of tests should be run during the implementation phase as well as later on when the site is fully functional. Even if the test was created during the implementation phase, it should still be run every time with every new release that goes out. It can even be used as a part of a stand-alone test set for the site. This helps in reducing our effort by huge amount as fas as testing is considered. We wrote feature files to test accross all our units, which carried out steps to navigate through our system through the outside views, i.e. they basically tested our units through \emph{Black Box Testing}. Then we wrote test cases for integration testing i.e. to check whether the units are interacting properly or not. For special cases like cpi,spi calculation: we wrote special test cases (on function level) to verify their functionality. This whole testing and code process helped us in saving time and debugging our code. Also, it helped in removing the manual testing of our software which needs to be done in any case.\\
Maintaining any software and test automation will definitly benefit from good development process.So we used git as our version control system. It will enable us in maintaining the current version of the automation and also in tracking the changes done to it and also to revert back to an earlier state if we get stuck into a problem.

\begin{thebibliography}{9}

\bibitem{github}
  https://github.com/pankajmore/oars

\bibitem{heroku}
  oars.herokuapp.com

\end{thebibliography}
\end{document}


